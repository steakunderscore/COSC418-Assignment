% COSC418 Interim Presentation
% Regan Gunther and Henry Jenkins - 2011

%\documentclass[handout]{beamer}
\documentclass{beamer}

\usetheme[secheader]{Boadilla}
\usecolortheme{seahorse}
\usepackage[latin1]{inputenc}
\usepackage{mathtools}

\title{COSC418 Project:\\Load balancing in a CTP based network}
\author{Henry Jenkins and Regan Gunther}
\date{September 16, 2011}
\institute[2011]{Department of Computer and Electrical Engineering,\\
    University of Canterbury, \\ Christchurch, \\ New Zealand}

\begin{document}

\frame{\titlepage}

%\section[Outline]{}
%\frame{\tableofcontents}

\section{CTP Load Balancing}
\subsection{Overview}

\begin{frame} 
  \frametitle{Overview}
  \begin{itemize}
    \item CTP - Collection Tree Protocol
      \begin{itemize}
        \item Address-free
        \item Best effort delivery
        \item Single hop transmissions
        \item Designed for low traffic environments
        \item Expected transmission (ETX) 
        \item Each node contains a Link Cost metric.
      \end{itemize}
    \item CTP LB - Collection Tree Protocol Load Balancing
      \begin{itemize}
        \item Add load balancing
        \item Avoid node hotspots
        \item Traffic balancing
      \end{itemize}
  \end{itemize}
\end{frame}


\subsection{Our Design}

\begin{frame}
  \frametitle{Our Design}
  \begin{itemize}
    \item $p^* = \text{arg min}  p \in \{\ \text{direct neighbours} \}\
      [\alpha  \cdot (ETX_{s,p} + ETX_p) + \beta \cdot L_p^s]$

    \item Our design is to keep $L_p^s$ locally 
    \item We increment $L_p^s$ as:
    \item $L_p^s = n_p^s - t_p^s$
    \item $n_p^s \geq t_p^s$
    \item Where $t_p^s$ is incremented periodically through a timer
    \item $t_p^s$ is included to stop $L_p^s$ getting too large.
    \end{itemize}
    Why do this instead of transmitting link load data
    \begin{itemize}
    \item Allows us to use the unmodified CTP routing packet
    \item Means we can add load balancing to an existing CTP network
    \item No extra data transmission
  \end{itemize}

\end{frame}

\subsection{Drawbacks}

\begin{frame}
  \frametitle{Drawbacks}
    What if $ETX_p + ETX_{p,s} + L_p^s$ (i.e. $ETX_s$) increases to be $\geq 2^{16}$?
  \begin{itemize}
    \item Bad things\ldots
    \item This is managed by including the $t_p^s$ in the loading parameter
    $L_p^s$
    \item (Remember $L_p^s = n_p^s - t_p^s$ )
    \item Periodic timer.
  \end{itemize}
\end{frame}  



\begin{frame}
  \frametitle{Drawbacks}
    ETX changing with each packet transmission causes routing updates all the
    time. All these updates would be inefficient.
  \begin{itemize}
    \item A solution to this is to only update routes when ETX changes by a given
    threshold
    \end{itemize}
\end{frame}  


\begin{frame}
  \frametitle{Drawbacks}
    Exponential ETX growth of a node
  \begin{itemize}
    \item In a path of 5 nodes to the root, $L_p^s$ increases with every packet.
    The more hops to the root, the more pronounced this effect
  \end{itemize}
\end{frame}  

\subsection{Conclusion}

\begin{frame}
  \frametitle{Conclusion}
  \begin{itemize}
    \item No results as yet
    \item Currently implementing and testing code
  \end{itemize}
\end{frame}  





%
%\subsection{Design Requirements}
%\frame {
%  \frametitle{Requirements}
%  \begin{enumerate}
%    \item<1-> Measurements
%    \begin{enumerate}
%	  \item Measure supply currents ranging from 0.1 uA up to 100 mA
%      \item Measure currents at Frequency range of DC to 10 kHz
%      \item Measure supply voltages of up to 10V with a resolution of 0.1 mV
%    \end{enumerate}
%    \item<2-> External Interface
%    \begin{enumerate}
%      \item Allows the user to turn sensor on or off, and to start or stop data
%            recording
%      \item Provide indication of on/off status and whether recording is underway
%      \item Allow retrieval of recorded measurements
%    \end{enumerate}
%    \item<3-> Power Supply (Battery)
%    \begin{enumerate}
%      \item Should be replaceable, or rechargeable via USB
%      \item Should allow at least 24 hours of use before requiring replacement or
%            recharge
%    \end{enumerate}
%  \end{enumerate}
%}
%
%\section{Technical}
%%TODO talk about requirments
%%TODO How are we measureing currnet and voltage
%\subsection{Main Circuit}
%\begin{frame}
%  \frametitle{Circuit Block Diagram}
%  \begin{center}
%    \includegraphics[width=.8\textwidth]{../Report/Images/CircuitBlockDiagram}
%  \end{center}
%\end{frame}
%
%\subsection{Data collection}
%\begin{frame}
%  \frametitle{ADC Circuits}
%    \begin{columns}[c] % the "c" option specifies center vertical alignment
%      \column{.5\textwidth} % column designated by a command
%        \begin{itemize}
%          \item<1-> MOSFET to disconnect
%          \item<1-> Voltage divider for 3.3$V_{Max}$
%        \end{itemize}
%      \column{.5\textwidth}
%        \begin{center}
%          \includegraphics[width=\textwidth]<1->{../Report/Images/VoltageSensorCircuit}
%        \end{center}
%    \end{columns}
%
%    \begin{columns}[c] % the "c" option specifies center vertical alignment
%      \column{.5\textwidth} % column designated by a command
%        \begin{itemize}
%          \item<2-> 1 Ohm Resister
%          \item<2-> $.1\mu V < V_{R} < 100mV$
%          \item<2-> Op-amp to bring voltage $0V < V_{opamp} < 3.3V$
%        \end{itemize}
%      \column{.5\textwidth}
%        \begin{center}
%          \includegraphics[width=\textwidth]<2->{../Report/Images/CurrentSensorCircuit}
%        \end{center}
%    \end{columns}
%\end{frame}
%
%\subsection{Data numbers}
%\frame {
%  \frametitle{How much data?}
%    \begin{itemize}
%      \item<1-> Measure current at frequency of 10 kHz
%        \begin{itemize}
%          \item That's 20 kHz to prevent aliasing
%        \end{itemize}
%      \item<2-> 20 bits per sample
%      \item<3-> 24 Hours
%      \item<4-> 4.32 GB
%    \end{itemize}
%}
%
%\subsection{Power Supply}
%\frame {
%  \frametitle{Batter Power Supply}
%  \begin{columns}
%  \column{.5\textwidth}
%    \begin{itemize}
%    \item<1-> Requires
%      \begin{itemize}
%        \item 3.3V
%        \item 335mA
%        \item 8 Amp-hours
%        \item Rechargeable
%      \end{itemize}
%    \item 3x D cell NiMH
%    \begin{itemize}
%    \item 9 Amp-hours
%    \item 3.6V
%    \end{itemize}
%  \end{itemize}
%  \column{.5\textwidth}
%    \includegraphics[width=\textwidth]{Images/Battery}
%  \end{columns}
%}
%
%\section{Conclusion}
%\subsection{Future Developments}
%\frame {
%  \frametitle{Future Developments}
%    \begin{itemize}
%      \item<1-> Event Marking
%      \item<2-> Collect Other Environmental Data
%      \begin{itemize}
%        \item Temperature
%        \item Humidity
%        \item Vibration
%        % On a pole causing antina shaking etc.
%        \item Moisture
%      \end{itemize}
%      \item<3-> It's own wireless sensor network
%    \end{itemize}
%}
%\subsection{Conclusion}
%\frame {
%  \frametitle{Conclusion}
%    \begin{itemize}
%      \item<1-> 24 Hours data collection
%      \item<2-> 4.3 GB of data collected
%      \begin{itemize}
%        \item Easily accessible by operator
%        \item Post processing on computer
%      \end{itemize}
%      \item<3-> Longer data collection periods
%    \end{itemize}
%}
%
%\section{Questions}
%
%\frame {
%  \frametitle{Questions?}
%  \begin{LARGE}
%    \begin{center}
%    Questions?
%    \end{center}
%  \end{LARGE}
%}


\end{document}
